%%%%%%%%%%%%%%%%%%%%%%%%%%%%%%%%%%%%%%%%%
% Medium Length Professional CV
% LaTeX Template
% Version 2.0 (8/5/13)
%
% This template has been downloaded from:
% http://www.LaTeXTemplates.com
%
% Original author:
% Rishi Shah 
%
% Important note:
% This template requires the resume.cls file to be in the same directory as the
% .tex file. The resume.cls file provides the resume style used for structuring the
% document.
%
%%%%%%%%%%%%%%%%%%%%%%%%%%%%%%%%%%%%%%%%%

%----------------------------------------------------------------------------------------
%	PACKAGES AND OTHER DOCUMENT CONFIGURATIONS
%----------------------------------------------------------------------------------------

\documentclass[letterpaper]{resume} % Use the custom resume.cls style

\usepackage{enumitem}% http://ctan.org/pkg/enumitem
\usepackage[left=0.75in,top=1in,right=0.75in,bottom=0.6in]{geometry} % 
%Document margins
\newcommand{\tab}[1]{\hspace{.2667\textwidth}\rlap{#1}}
\newcommand{\itab}[1]{\hspace{0em}\rlap{#1}}
\name{Chong-Chong He} % Your name
\address{\small UMD-Astronomy, 1113 PSC Bldg. 415, College Park, MD 20742} % Your address
%\address{123 Pleasant Lane \\ City, State 12345} % Your secondary addess (optional)
\address{(+1) 240-413-9772 \\ chongchong@astro.umd.edu} % Your phone number and email

\newcommand{\myspace}{\vspace{2mm}}

\begin{document}

%----------------------------------------------------------------------------------------
%	EDUCATION SECTION
%----------------------------------------------------------------------------------------

\begin{rSection}{Education}

{\bf University of Maryland} \\ 
M.S. Astronomy \hfill {\em 2016 -- 2018} \\
  \vspace{-1mm}
\hspace{6mm} \emph{\small Research Topic: 
Simulating Star Clusters Across Cosmic Time}\\
%Simulating Star Formation in Moleculer Clouds: Ionizing Photon Escape Fraction}\\
Ph.D. Astronomy \hfill {\em 2018 -- Present} \\
\vspace{-1mm}
\hspace{6mm} \emph{\small Thesis Topic (Planed): Simulating Compact Star Clusters and Growth of the Seed Black Holes in the First} \\
\vspace{-1mm}
\hspace{6mm} \emph{\small Galaxies}
\item {\bf Jilin University} \hfill {\em 2012 -- 2016} 
\\ B.S. Physics, with Highest Honor %\hfill {Overall Percentage: $98.6\%$}
\item {\bf Georgia Institute of Technology} \hfill {\em 1/2015 --
    7/2015}
\\ Visiting Honors Student Program, Language and Physics
%Minor in Linguistics \smallskip \\
%Member of Eta Kappa Nu \\
%Member of Upsilon Pi Epsilon \\

\end{rSection}

% \begin{rSection}{Education}

% {\bf University of Maryland} 
% \\ M.S. Astronomy \hfill {\em 2016 -- 2018} 
% \\ Ph.D. Astronomy \hfill {\em 2018 -- Present}
% \item {\bf Jilin University} \hfill {\em 2012 -- 2016} 
% \\ B.S. Physics, with Highest Honor %\hfill {Overall Percentage: $98.6\%$}
% \item {\bf Georgia Institute of Technology} \hfill {\em 1/2015 --
%     7/2015}
% \\ Visiting Honors Student Program
% %Minor in Linguistics \smallskip \\
% %Member of Eta Kappa Nu \\
% %Member of Upsilon Pi Epsilon \\

% \end{rSection}

%--------------------------------------------------------------------------------
%    Projects And Seminars
%-----------------------------------------------------------------------------------------------
% \begin{rSection}{Projects}
% {\bf Dynamic Analysis of Buckling Restrained Braces}\\
% The project aims at designing and fabrication of two Buckling
% Restrained Braces which were analyzed under dynamic loading. As

% \myspace
% {\bf Microtunneling}\\
% Presented a seminar on Micro Tunneling, explaining its advantages over conventional method of drainage laying systems. Analysis considering direct and indirect cost of micro tunneling was also discussed.

% \end{rSection}

% %----------------------------------------------------------------------------------------
% %	TECHNICAL STRENGTHS SECTION
% %----------------------------------------------------------------------------------------

% \begin{rSection}{Technical Strengths}

% \begin{tabular}{ @{} >{\bfseries}l @{\hspace{6ex}} l }
% Modeling and Analysis \ & AutoCad, Revit, StaadPro \\
% Software \& Tools & MS Office, Latex \\
% \end{tabular}

% \end{rSection}

%----------------------------------------------------------------------------------------
%	AWARDS
%----------------------------------------------------------------------------------------

\begin{rSection}{HONORS \& AWARDS} \itemsep -3pt
{\bf China Youth Science and Technology Innovation Award}, 2016
\item {\bf Dean's Honored Graduates, Jilin University}, 2016
  \vspace{-1mm}
\item \hspace{6mm} \emph{\small The highest honor awarded to graduating
    seniors in the College of Physics}
\item {\bf Tang-Ao Qing Supreme Award for Excellence in Research \&
  Practice}, 2016
\item {\bf China Scholarship Council Scholarship for Overseas Study},
  2014
  \vspace{-1mm}
  \item \hspace{6mm} \emph{\small Awarded to the top 1\% in the College of
    Physics, Jilin University}

\end{rSection}

%----------------------------------------------------------------------------------------
%	Publications
%----------------------------------------------------------------------------------------

\begin{rSection}{PUBLICATIONS} \itemsep -3pt
  {\bf He, C.-C.}, Ricotti, M., \& Geen, S., ``Simulating Star Clusters Across Cosmic Time: II. Fraction of Ionizing Photons Escaping from Molecular Clouds'', submitted to {\it Monthly Notices of the Royal Astronomical Society}.
  \item {\bf He, C.-C.}, Ricotti, M., \& Geen, S. 2019, ``Simulating star clusters across cosmic time - I. Initial mass function, star formation rates, and efficiencies'', {\it Monthly Notices of the Royal Astronomical Society}, 489, 1880-1898.
  \item {\bf He, C.-C.} \& Keek, L. 2016, ``Anisotropy of X-Ray Bursts from Neutron Stars with Concave Accretion Disks'', {\it The Astrophysical Journal}, 819, 47.
\end{rSection}

%----------------------------------------------------------------------------------------
%	EXAMPLE SECTION
%----------------------------------------------------------------------------------------

\begin{rSection}{SUCCESSFUL PROPOSALS}
{\bf MARCC/Bluecrab Supercomputer}, Q1 2018, 200 kSU monthly allocation
\end{rSection}

%----------------------------------------------------------------------------------------
% Extra Curricular
%----------------------------------------------------------------------------------------

\begin{rSection}{TEACHING EXPERIENCE} \itemsep -3pt
\renewcommand\labelitemi{\tiny$\bullet$}
% \renewcommand\labelitemi{{\boldmath$\cdot$}}
% {\bf Teaching Assistant} \hfill {\em 08/2017 -- 12/2017}\\
%   \emph{University of Maryland}
%   \begin{itemize}[noitemsep,topsep=-2pt]
%   \item Course: ASTR 100: \emph{Introduction to Astronomy}
%   \item Leading 2 discussion sections per week
%   \item Grading and holding office hours to provide additional
%     guidance to students
%   \end{itemize}
% \vspace{2mm}

{\bf Teaching Assistant} \hfill {\em 08/2016 -- 05/2018}\\
  \emph{University of Maryland}
  \begin{itemize}[noitemsep,topsep=-2pt]
  \item Courses: ASTR 100, ASTR 420, ASTR 330, ASTR 300, ASTR 340
  \item Led 2 discussion sections per week (ASTR 100, 08/2017 -- 12/2017) 
  \item Graded worksheets, homework, and exams.
  \item Held office hours to provide additional guidance to students
  \end{itemize}
\end{rSection}

\begin{rSection}{COMPUTER SKILLS}
Programming and software fluency in {\em Python, C/C++, Fortran, \LaTeX}
\item Experience with openMP and MPI parallel programming
\item Experience with {\em Mathematica, MATLAB, HTML/CSS}
\end{rSection}

\end{document}

%%% Local Variables:
%%% mode: latex
%%% TeX-master: t
%%% End:
